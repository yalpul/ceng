\documentclass[12pt]{article}
\usepackage[utf8]{inputenc}
\usepackage{float}
\usepackage{amsmath}
\hbadness=10000

\usepackage[hmargin=3cm,vmargin=6.0cm]{geometry}
%\topmargin=0cm
\topmargin=-2cm
\addtolength{\textheight}{6.5cm}
\addtolength{\textwidth}{2.0cm}
%\setlength{\leftmargin}{-5cm}
\setlength{\oddsidemargin}{0.0cm}
\setlength{\evensidemargin}{0.0cm}

\newcommand{\HRule}{\rule{\linewidth}{1mm}}

%misc libraries goes here
\usepackage{tabularx}
\usepackage{tikz}
\usetikzlibrary{positioning}
\renewcommand\tabularxcolumn[1]{m{#1}}
\newcolumntype{M}{>{\centering\arraybackslash}m{1.15cm}}
\newcommand\tikzmark[2]{%
\tikz[remember picture,baseline] \node[inner sep=2pt,outer sep=0] (#1){#2};%
}

\newcommand\link[2]{%
\begin{tikzpicture}[remember picture, overlay, >=stealth, shift={(0,0)}]
  \draw[->] (#1) to (#2);
\end{tikzpicture}%
}
\begin{document}

\noindent
\HRule \\[3mm]
\begin{flushright}

                                         \LARGE \textbf{CENG 465}  \\[4mm]
                                         \Large Introduction to Bioinformatics \\[4mm]
                                        \normalsize      Spring '2019-2020 \\
                                           \Large   Homework 1 \\
                    \normalsize Deadline: March 1, 23:59 \\
                    \normalsize Submission: via ODTUCLASS
\end{flushright}
\HRule

\section*{Student Information }
Full Name : Yasin Fatih ALPUL \\
Id Number :  2098739 \\

\section*{Problem 1}
\begin{figure}[H]
    \centering
\begin{tabularx}{\textwidth}{|M|M|M|M|M|M|M|M|M|M|M|}
\cline{1-11}
	&-	&M	&I	&M	&A	&G	&E	&D	&I	&L	\\ \cline{1-11}
-	&0	&0	&0	&0	&0	&0	&0	&0	&0	&0	\\\cline{1-11}
G	&0	&0	&0	&0	&0	&7	&3	&0	&0	&0	\\\cline{1-11}
A	&0	&0	&\tikzmark{h}{0}	&0	&7	&3	&1	&0	&0	&0	\\\cline{1-11}
M	&0	&7	&3	&\tikzmark{f}{7}	&3	&1	&0	&0	&0	&0	\\\cline{1-11}
A	&0	&3	&1	&3	&\tikzmark{e}{14}	&\tikzmark{c}{10}	&6	&2	&0	&0	\\\cline{1-11}
E	&0	&0	&0	&0	&10	&8	&\tikzmark{b}{17}	&13	&9	&5\\\cline{1-11}
D	&0	&0	&0	&0	&6	&4	&13	&\tikzmark{a}{24}	&20	&16	\\\cline{1-11}
K	&0	&0	&0	&0	&2	&0	&9	&20	&18	&14\\\cline{1-11}
\end{tabularx}
\link{a}{b}
\link{b}{c}
\link{c}{e}
\link{e}{f}
\link{f}{h}
    \caption{Table for Problem 1}
\end{figure}
\begin{align*}
    \text{Match score} &= 7 \\
    \text{Mismatch penalty} &= -6\\
    \text{Gap penalty} &= -4 \\
    \text{Score of local alignment} &= 24
\end{align*}
If we look at intersection of the second row and the sixth column, 
    namely the intersection of $G$s, it became $7$ from $0$ as a result 
    of a match. Thus, a match score is $7$.
 
 If we look at the cell at immediately right of that cell, the intersection 
    of $G$ and $E$, we see a score of $3$. Since $G$ and $E$ does not 
    match, this must be a result of a gap. The score became $3$ from
    $7$. Thus, the gap penalty is $-4$.
 
 If we look at the cell at immediately under the previous cell, the
    intersection of $A$ and $E$, we see a score of $1$. Since $A$ and 
    $E$ does not match, this must be coming from either a mismatch or 
    a gap. The adjacent cells have the value $3$. So, if it would have 
    been a gap, the score should be $-1$ and as a result $0$ because 
    we use local alignment and $0>-1$. So the score must be coming 
    from a mismatch. The previous cell is a $7$ and thus a mismatch 
    penalty is $-6$.
    
Since the greatest value of the table is $24$, the score of the 
    local alignment is $24$.
    
The best local alignments is:
\begin{verbatim}
    MAGED
    MA-ED
\end{verbatim}
If we calculate the alignment, there are four matches ($4\times 7$) and one 
    gap ($-4$), resulting in $4\times7-4=24$. Thus, we get the same value as 
    the traceback of cells from the table.
\section*{Problem 2}

\begin{figure}[H]
    \centering
\begin{tabularx}{\textwidth}{|M|M|M|M|M|M|M|M|M|M|M|}
\cline{1-11}
	&-	&M	&C	&G	&M	&G	&C	&M	&E	&L	\\ \cline{1-11}
-	&\tikzmark{k}{0}	&\tikzmark{j}{-4}	&\tikzmark{i}{-8}	&-12	&-16	&-20	&-24	&-28	&-32	&-36	\\\cline{1-11}
G	&-4	&-3	&-7	&\tikzmark{h}{-2}	&-6	&-10	&-14	&-18	&-22	&-26	\\\cline{1-11}
M	&-8	&1	&-3	&-6	&\tikzmark{g}{3}	&\tikzmark{f}{-1}	&-5	&-9	&-13	&-17	\\\cline{1-11}
C	&-12	&-3	&10	&6	&2	&0	&\tikzmark{e}{8}	&4	&0	&-4	\\\cline{1-11}
M	&-16	&-7	&6	&7	&11	&7	&4	&\tikzmark{d}{13}	&9	&5	\\\cline{1-11}
E	&-20	&-11	&2	&4	&7	&9	&5	&9	&\tikzmark{c}{18}	&14\\\cline{1-11}
D	&-24	&-15	&-2	&1	&3	&6	&6	&5	&\tikzmark{b}{14}	&14	\\\cline{1-11}
L	&-28	&-19	&-6	&-3	&3	&2	&5	&8	&10	&\tikzmark{a}{18}\\\cline{1-11}
\end{tabularx}
\link{a}{b}
\link{b}{c}
\link{c}{d}
\link{d}{e}
\link{e}{f}
\link{f}{g}
\link{g}{h}
\link{h}{i}
\link{i}{j}
\link{j}{k}
    \caption{Table for Problem 2}
\end{figure}
The best alignment is:
\begin{verbatim}
    MCGMGCME-L
    --GM-CMEDL
\end{verbatim}
If we calculate the scores in the final alignment, we get four gaps 
    ($4\times-4$) and the scores from the table (GG$=6$, MM$=5$, CC$=9$, 
    MM$=5$, EE$=5$, LL$=4$). \\

Total score is $6+5+9+5+5+4+4\times-4=18$, 
    which is the same value as the table we created.
\end{document}
