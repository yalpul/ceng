\documentclass[12pt]{article}

\author{ALPUL, Yasin Fatih (e2098739)}
\title{Homework 1 - Design Document}

\begin{document}
\maketitle

\section{Overall Design}
The following are the classes used in the design:
\begin{itemize}
  \item Bullet
  \item Commando
  \item FastZombie
  \item Position
  \item RegularSoldier
  \item RegularZombie
  \item SimulationController
  \item SimulationObject
  \item SimulationRunner (driver)
  \item SlowZombie
  \item Sniper
  \item Soldier
  \item SoldierState
  \item SoldierType
  \item Zombie
  \item ZombieState
  \item ZombieType
\end{itemize}

The \texttt{SimulationObject} is the abstract base class for
  \texttt{Soldier}, \texttt{Zombie} and \texttt{Bullet} classes.
  Moreover, \texttt{Soldier} is the abstract base class for
  \texttt{RegularSoldier}, \texttt{Commando} and \texttt{Sniper}
  classes whereas \texttt{Zombie} is the abstract base class for
  \texttt{RegularZombie}, \texttt{SlowZombie} and
  \texttt{FastZombie} classes.

In \texttt{SimulationController} class, adding new objects is
  done by checking whether the object is of which derived type
  of \texttt{SimulationObject}. Then, it adds the object to the
  relevant \texttt{List}. Removing is similar.

For simulating a single step, the order is chosen to be as the
  following:
  \begin{enumerate}
    \item Simulate all soldiers
    \item Simulate all zombies
    \item Simulate all bullets
  \end{enumerate}
  Later, necessary operations are taken care of such as clearing
  the battlefield from dead objects.

For storing the objects in the simulation, bullets are chosen to be
  stored in \texttt{ArrayList} container types, whereas soldier
  and zombie types are stored in \texttt{LinkedList} containers.
  The reason for that is, since there will be a possibility that
  a soldier/zombie can be removed after some step in simulation,
  an efficient remove operation will be needed. One cannot anticipate
  which soldier/zombie will be removed next. Therefore, in which
  exact position in the container will be removed cannot be known.
  So in this scenario, a \texttt{LinkedList} would be more efficient
  than an \texttt{ArrayList}. Bullets are stored in \texttt{ArrayList}
  since the list is completely replaced after each step in the
  simulation.


\section{Design Principles}

\subsection{Minimizing the Accessibility}
Data fields of classes have been made \texttt{private} where
  possible. The reason for that is to not expose the internal
  details of the objects. For instance, \texttt{shootingRange}
  and \texttt{collisionRange} fields have been made private
  and public methods have been provided using those fields.
  However, note that no direct setters/getters have been provided.
  Instead, to improve abstaction, high level methods have been
  provided. For instance, zombie types have a method with the
  signature \texttt{boolean canDetect(Soldier)}, which internally
  uses its \texttt{detectionRange}. Additionally, in the program
  code, expressions that are very similar to natural language
  can be written such as \texttt{if (this.canDetect(closestSoldier))}.
  This design improves abstraction.

\subsection{Open/Closed Principle}
This principle states that, as covered in class, \emph{an entity
  should be open for extension, yet closed for modification}.
  This principle can be honored by using abstraction, polymorphism
  and inheritance. As it is also covered in the next section,
  it is achieved by using polymorphic types instead of \texttt{enum}
  types for soldier/zombie types.

\subsection{The Single Choice Principle}
This principle, as covered in class, states that \emph{whenever
  a software system must support a set of alternatives, ideally
  only one class in the system knows the entire set of alternatives}.
  This principle is honored in the sense that, only
  \texttt{SimulationController} class knows about the entire set of
  soldier/zombie classes. Even then, it doesn't know about the classes
  which are derived from soldier/zombie abstract base classes. When
  a \texttt{SimulationObject} is added, it only checks for whether
  the object is a \texttt{Soldier}, \texttt{Zombie} or a
  \texttt{Bullet}.

\subsection{The Liskov Substitution Principle}
This principle states that, as covered in class, \emph{functions
  that use references to base classes must be able to use objects
    of derived classes without knowing it}. This principle is honored
    in the sense that, in the \texttt{Zombie} class, there are a few
    methods that take a \texttt{Soldier} reference and call methods
    of it. However, \texttt{Soldier} is an abstract base class so
    there is no actual \texttt{Soldier} type whatsoever. In reality,
    it is either a \texttt{RegularSoldier}, a \texttt{Commando} or a
    \texttt{Sniper}. But it doesn't know that, it is able to use
    the reference of the base class without knowing it.

\section{Zombie and Soldier Types}
In the implementation, \textit{enum} types of zombies and soldiers
  are not used. Instead, virtual method invocation facilities
  of JVM have been favored. There are several reasons for that.
  \begin{enumerate}
    \item \textbf{Extendability:} In the case of a need
      for addition of a new soldier/zombie type, a number of
      changes would be needed to be made to source files.
      Assuming a check for the type of a soldier/zombie,
      additional checks for new types would be added.
      Also, the \textit{enum} types would be recompiled.
      In this way however, new types would be just plugged and it
      will work.
    \item \textbf{Abstraction:} Soldier/zombie type is an
      internal detail. Users will only need to know whether it
      is a soldier/zombie or not. User will need a generic soldier
      object and JVM will call the right method according to its
      type using \texttt{invokevirtual} bytecode.
  \end{enumerate}

\section{Javadoc Style}
While writing javadoc comments, the following conventions have
  been followed.\\
  Official Oracle Documentation:\\
  \texttt{https://www.oracle.com/technetwork/java/javase/documentation/index-137868.html}

\end{document}
